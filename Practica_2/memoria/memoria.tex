\documentclass[12pt, a4paper, spanish]{article}
\usepackage[spanish]{babel}
\usepackage[utf8]{inputenc}
\usepackage{setspace}
\usepackage[
pdftex,
pdfauthor={--- Arturo Vidal Peña ---},
pdftitle={--- Práctica 2 de Prolog ---},
hidelinks]{hyperref}

\usepackage{amssymb}
\usepackage{amsmath}
\usepackage{txfonts}
\usepackage{mathdots}
\usepackage{graphicx}
\usepackage{float}

% --- IMAGES ---
\usepackage{subfig}
\DeclareGraphicsExtensions{.png,.jpg,.pdf,.mps,.gif,.bmp}
% --- IMAGES ---

% --- MARGIN DIMENSIONS ---
\frenchspacing \addtolength{\hoffset}{-1.5cm}
\addtolength{\textwidth}{3cm} \addtolength{\voffset}{-2.5cm}
\addtolength{\textheight}{4cm}
\setlength{\headheight}{15pt}
% --- MARGIN DIMENSIONS ---

% --- CODE ---

\usepackage{listings}
\usepackage[usenames,dvipsnames]{color}

\definecolor{codegreen}{rgb}{0,0.6,0}
\definecolor{codegray}{rgb}{0.5,0.5,0.5}
\definecolor{codepurple}{rgb}{0.58,0,0.82}
\definecolor{backcolour}{rgb}{0.95,0.95,0.92}
\definecolor{predicate}{RGB}{250,202,43}


\lstdefinelanguage{Ciao-Prolog}{
	%keyword3
	classoffset = 2,
	morekeywords = {A, B, Comp, M, X, Y, Term1, Aridad1, Term2, Aridad2, Orden, ListaX, ListaY, Lista, Hojas, T, T1, T2, H, Arg1, Brg1, Arbol, Hoja, Tree,
		RaizIzq, Raiz, OrdenF, OrdenS, Element, Left, ElementRight, ListaSalida},
	keywordstyle = \color{codegreen},	
	classoffset = 3,
	morekeywords = {arg, functor, call, var, nonvar, append},
	keywordstyle = \color{blue},	
	classoffset = 0,
	sensitive = true,
	morecomment = [l]{\%},
	morecomment = [s]{/*}{*/},
	commentstyle = \color{gray},
	morestring = [b]",
	morestring = [b]',
	stringstyle = \color{purple}
}
\lstset{
	language={Ciao-Prolog},
	basicstyle={\small},
	identifierstyle={\small},
	commentstyle={\small\itshape},
	keywordstyle={\small\bfseries},
	ndkeywordstyle={\small},
	stringstyle={\small\ttfamily},
	frame={tb},
	breaklines=true,
	columns=[l]{fullflexible},
	numbers=left,
	xrightmargin=0em,
	xleftmargin=3em,
	numberstyle={\scriptsize},
	stepnumber=1,
	numbersep=1em,
	lineskip=-0.5ex,
}


\lstdefinestyle{mystyle}{
	backgroundcolor=\color{backcolour},   
	commentstyle=\color{codegreen},
	keywordstyle=\color{blue},
	numberstyle=\tiny\color{codegray},
	stringstyle=\color{codepurple},
	basicstyle=\footnotesize,
	breakatwhitespace=false,         
	breaklines=true,                 
	captionpos=b,                    
	keepspaces=true,                 
	numbers=left,                    
	numbersep=5pt,                  
	showspaces=false,                
	showstringspaces=false,
	showtabs=false,                  
	tabsize=2
}

\lstset{style=mystyle}

% --- CODE ---

% --- TOC DOTS ---
\usepackage{tocloft}
\renewcommand{\cftsecleader}{\cftdotfill{\cftdotsep}}
% --- TOC DOTS ---

% --- TITLE DATA ---
\title{\textbf{Práctica 2: Programación en Prolog} \\
	\textsc{Programación Declarativa: Lógica y Restricciones} \\
	\emph{DIA}}
\author{\emph{Vidal Peña, Arturo}\\
		\emph{W140307}}
\date{\underline{\today}}
% --- TITLE DATA ---

\begin{document}
	
% --- TITLE ---
\maketitle
\thispagestyle{empty}
\pagenumbering{gobble}
\renewcommand*\contentsname{Índice de contenidos}
\tableofcontents
\pagebreak
% --- TITLE ---

\pagenumbering{arabic}

\section{Código empleado y las explicaciones}

\subsection{Predicado menor/4}
\lstinputlisting[language=Ciao-Prolog]{codeFiles/menor.pl}
En este predicado, preparo con \emph{functor/3} una estructura en \emph{X} de tres términos, y añado \emph{Comp, A} y \emph{B} con los predicados arg/3.\\
Luego, con \emph{call/1} llamo a esa estructura y guardo el resultado en \emph{M}.

\subsection{Predicado menor\_o\_igual/4}
\lstinputlisting[language=Ciao-Prolog]{codeFiles/menor_o_igual.pl}


\subsection{Predicado lista\_hojas/2}
\lstinputlisting[language=Ciao-Prolog]{codeFiles/lista_hojas.pl}


\subsection{Predicado hojas\_arbol/3}
\lstinputlisting[language=Ciao-Prolog]{codeFiles/hojas_arbol.pl}


\subsection{Predicado ordenacion/3}
\lstinputlisting[language=Ciao-Prolog]{codeFiles/ordenacion.pl}


\subsection{Predicado ordenar/3}
\lstinputlisting[language=Ciao-Prolog]{codeFiles/ordenar.pl}


\newpage

\section{Pruebas realizadas}

Todas las pruebas a continuación han dado los valores esperados.

\begin{enumerate}
	\item \underline{nat/1}:
	\begin{itemize}
	\item nat(0),            
	\item nat(s(0)),
	\item nat(s(s(s(s(0))))),
	\item nat(t)
	\end{itemize}
	
	\item \underline{menor\_igual/2}:
	\begin{itemize}
	\item menor\_igual(s(0),s(0)),             
	\item menor\_igual(s(s(0)),s(s(0))),
	\item menor\_igual(s(0),s(s(0)))
	\end{itemize}

\item \underline{resta/2}:
\begin{itemize}
	\item resta(s(0),0,s(0)),            
	\item resta(s(0),s(0),0),             
	\item resta(s(s(s(s(0)))),s(s(s(s(0)))),0)
\end{itemize}

\item \underline{par/1}:
\begin{itemize}
	\item par(0)             
	\item par(s(s(0))),             
	\item par(s(s(s(s(0)))))
\end{itemize}

\item \underline{iguales/2}:
\begin{itemize}
	\item iguales(s(0),s(0)),             
	\item iguales(s(0),s(0)),             
	\item iguales(s(s(s(s(0)))),s(s(s(s(0))))),
\end{itemize}

\item \underline{color/1}:
\begin{itemize}
	\item color(r),             
	\item color(v),             
	\item color(am),
	\item color(a),
	\item color(b)
\end{itemize}

\item \underline{esTorre/1}:
\begin{itemize}
	\item esTorre([pieza(s(0),s(s(0)),s(0),r)]),             
	\item esTorre([pieza(s(0),s(s(0)),s(0),r),pieza(s(0),s(0),s(0),a)]),             
	\item esTorre([pieza(s(s(s(0))),s(s(s(0))),s(s(s(0))),am),pieza(s(s(s(0))),s(s(s(0))),s(s(s(0))),v)])
\end{itemize}

\item \underline{alturaTorre/2}:
\begin{itemize}
	\item alturaTorre([pieza(s(0),s(0),s(0),r)],s(0)),             
	\item alturaTorre([pieza(s(0),s(0),s(0),r),pieza(s(s(0)),s(s(0)),s(0),a)],s(s(s(0)))),   
	\item alturaTorre([pieza(s(0),s(0),s(0),r),pieza(s(0),s(0),s(0),r),pieza(s(0),s(s(0)),s(0),r)],s(s(s(s(0)))))
\end{itemize}

\item \underline{coloresTorre/2}:
\begin{itemize}
	\item coloresTorre([pieza(s(0),s(0),s(0),r)],[r]),             
	\item coloresTorre([pieza(s(0),s(0),s(0),r),pieza(s(0),s(s(0)),s(0),a)],[r,a]),             
	\item coloresTorre([pieza(s(0),s(s(0)),s(0),am),pieza(s(0),s(0),s(0),r),pieza(s(0),s(s(0)),s(0),a)],[am,r,a])
\end{itemize}

\item \underline{coloresIncluidos/2}:
\begin{itemize}
	\item coloresIncluidos([pieza(s(0),s(0),s(0),r)],[pieza(s(0),s(0),s(0),r)]),             
	\item coloresIncluidos([pieza(s(0),s(0),s(0),a),pieza(s(0),s(0),s(0),a)],[pieza(s(s(0)),s(0),s(s(0)),a)]),             
	\item coloresIncluidos([pieza(s(0),s(0),s(0),r),pieza(s(0),s(0),s(0),v),pieza(s(0),s(0),s(0),a)])
\end{itemize}

\item \underline{esEdificioPar/2}:
\begin{itemize}
	\item esEdificioPar([[a,a],[v,v],[a,a],[v,v]]),             
	\item esEdificioPar([[a,a,r,v,am,r],[v,r,v,v,am,r],[v,r,v,r,a,r],[v,r,a,r,v,r],[v,r,a,r,v,am]]),             
	\item esEdificioPar([[a,a,am,r],[v,v,am,r],[v,r,a,r],[v,r,a,r],[v,r,a,am]])
\end{itemize}

%\item \underline{esEdificioPiramide/2}:
%\begin{itemize}
%	\item esEdificioPiramide([[b,b,b,r,b,b,b],[b,am,r,b,b,b,b]]),             
%	\item esEdificioPiramide([[b,am,b],[b,am,r]]),             
%	\item esEdificioPiramide([[b,b,v,b,b],[b,a,r,a,b]])
%\end{itemize}
\end{enumerate}

\section{Comentarios adicionales}
\end{document}

