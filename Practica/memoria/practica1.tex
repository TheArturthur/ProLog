% Generated by GrindEQ Word-to-LaTeX 
\documentclass{article} %%% use \documentstyle for old LaTeX compilers

\usepackage[english]{babel} %%% 'french', 'german', 'spanish', 'danish', etc.
\usepackage{amssymb}
\usepackage{amsmath}
\usepackage{txfonts}
\usepackage{mathdots}
\usepackage[classicReIm]{kpfonts}
\usepackage[dvips]{graphicx} %%% use 'pdftex' instead of 'dvips' for PDF output

% You can include more LaTeX packages here 


\begin{document}

%\selectlanguage{english} %%% remove comment delimiter ('%') and select language if required


\noindent 

\noindent 

\noindent 

\noindent 
\[2\] 


\noindent \textbf{\underbar{Práctica 1: Programación Lógica Pura.}}

\noindent \textbf{\underbar{}}

\noindent \underbar{Vidal Peña, Arturo}

\noindent \underbar{W140307}

\noindent 5 de abril de 2019

\noindent \eject 

\begin{enumerate}
\item  \underbar{Código empleado y las explicaciones}:\underbar{}
\end{enumerate}

\noindent 

\begin{enumerate}
\item  Definición de la base de hechos de los colores de las piezas de LEGO$\mathrm{{}^{TM}}$:
\end{enumerate}

\noindent Disponemos una base de hechos con los colores amarillo (am), verde (v), azul (a) y rojo (r) para hacer las comprobaciones de los colores de las piezas.

\noindent 

\begin{enumerate}
\item  Definición de predicados auxiliares comunes a los especificados en la práctica:
\end{enumerate}

\noindent Definimos una serie de predicados auxiliares: 

\noindent 

\begin{enumerate}
\item \begin{enumerate}
\item  nat/1: comprueba que el número pasado como argumento es un número natural. Partiendo de 0 (que definimos natural), un número X será natural si X-1 lo es.

\item  suma/3, que será cierto cuando el tercer argumento sea la suma aritmética de los dos primeros.

\item  resta/3, que será cierto cuando el tercer argumento sea la resta aritmética de los dos primeros.

\item  menor\_igual/2, que será cierto si el primer argumento es menor o igual que el segundo. Se llama recursivamente con el número de Peano anterior a cada argumento hasta que el primero sea 0 y el segundo, mayor que 0.

\item  mayor\_igual/2, su funcionamiento es igual que el anterior, sólo que comprueba si el primer argumento es mayor o igual que el segundo.

\item  iguales/2, será cierto cuando ambos argumentos pasados como números de Peano tengan el mismo valor. Se llama recursivamente comprobando que lleguen a 0 al mismo tiempo.

\item  par/1: será cierto si, restando 2 al número pasado por argumento recursivamente, se llega a 0.

\item  impar/1: igual que el anterior, sólo que comprobando que llega a 1.

\item  esPieza/4: comprueba si los argumentos \textit{Altura}, \textit{Anchura} y \textit{Profundidad} pasados como números de Peano son naturales, y si \textit{Color} está en la base de hechos.

\item  p/3: siendo pasada una lista como primer argumento, será cierto si el segundo argumento es la cabeza y el tercero, la cola de la lista.

\item  member/2: comprueba si el primer argumento se encuentra en la lista que se pasa en el segundo argumento.
\end{enumerate}
\end{enumerate}

\noindent 

\begin{enumerate}
\item  Definición del predicado esTorre/1:
\end{enumerate}

\noindent El caso base se afirma si la torre se compone de una única pieza, llamando a esPieza/1.

\noindent En el caso de tener más de una pieza, comprueba recursivamente que las piezas de la torre son válidas con esPieza/1 y que la pieza en cabeza es más pequeña en \textit{Anchura} y \textit{Profundidad} que la siguiente en la torre, llegando al caso base con la última pieza.

\noindent 

\begin{enumerate}
\item  Definición del predicado alturaTorre/2:
\end{enumerate}

\noindent El caso base se afirma si la torre de una única pieza es válida (esTorre/1), si el segundo parámetro en natural, y si este segundo parámetro vale 0 al ser restado por la altura de la pieza que conforma la torre.

\noindent En el caso de haber más de una pieza, comprueba que la lista es una torre válida, comprueba que el segundo parámetro es un natural, y llama al predicado auxiliar exclusivo de alturaTorre/2, pasando los mismos argumentos. Este predicado sacarAltura/2 va restando recursivamente al segundo parámetro las alturas de las piezas que conforman la torre, llegando al caso base con la última pieza.

\noindent 

\begin{enumerate}
\item  Definición del predicado coloresTorre/2:
\end{enumerate}

\noindent El caso base se afirma si la torre de una única pieza es válida (esTorre/1), y si el color de la pieza pertenece a la lista de colores pasada en el segundo parámetro.

\noindent En el caso de haber más de una pieza, comprueba si la lista es una torre válida, si el segundo parámetro es un número natural, y luego llama al predicado auxiliar exclusivo de coloresTorre/2, pasando los mismos argumentos. Este predicado sacarColores/2 comprueba recursivamente que los colores de las piezas de la torre pertenecen a la lista de colores del segundo parámetro, llegando al caso base con la última pieza.

\noindent 

\begin{enumerate}
\item  Definición del predicado coloresIncluidos/2, junto a sus auxiliares:
\end{enumerate}

\noindent Primero comprueba si ambas torres pasadas como parámetros son válidas, y luego llama al predicado auxiliar exclusivo de coloresIncluidos, pasando la primera torre en el primer parámetro y la segunda, en el segundo y tercero.

\noindent Este predicado (comprobarColores/3), compara recursivamente el primer color de Torre1 con los colores de Torre2. Cuando acaba, copia la Torre2 y vuelve a empezar con el siguiente color de Torre1.

\noindent 

\begin{enumerate}
\item  Definición del predicado esEdificioPar/1, junto a sus auxiliares:
\end{enumerate}

\noindent El caso base de esEdificioPar/1 es si la construcción sólo tiene una línea, en cuyo caso será cierto si esa línea es par, llamando al predicado lineaPar/1. En el caso de haber más de una línea, comprobará con lineaPar/1 si todas las líneas de la construcción son pares.

\noindent Este predicado auxiliar llama a longitudLinea/2, pasando la línea y una variable en la que unificar la longitud de esta. Este predicado llama a su vez a longitudSinBlanco/3, pasando la línea, inicializando N1 a 0 y la variable a unificar en N2. 

\noindent En longitudSinBlanco/1, si la cabeza de Linea es un blanco, se llama recursivamente con la cola de Linea como primer argumento. En otro caso, comprueba que la cabeza es un color válido, y se llama recursivamente con la cola como primer argumento, y el siguiente a N1 como segundo, siendo N2 siempre el tercero. Al acabar, cuando la línea está vacía, iguala N2 a N1.

\noindent  Al unificar N2 con el segundo argumento de longitudLinea/2, comprueba que sea un número par con par/1. Si es cierto, la línea es par.

\noindent 

\begin{enumerate}
\item  Definición del predicado esEdificioPirámide/1 y sus auxiliares:
\end{enumerate}

\noindent 


\end{document}

